\hypertarget{ntrxiqpar_8h}{
\section{include/ntrxiqpar.h File Reference}
\label{ntrxiqpar_8h}\index{include/ntrxiqpar.h@{include/ntrxiqpar.h}}
}
Utility functions for nano\-LOC transmit mode setting. 

{\tt \#include \char`\"{}config.h\char`\"{}}\par
{\tt \#include \char`\"{}ntrxtypes.h\char`\"{}}\par
\subsection*{Functions}
\begin{CompactItemize}
\item 
void \hyperlink{ntrxiqpar_8h_f755eef3bbb9b3f1990d8754679a5695}{NTRXSet\-Agc\-Values} (uint8\_\-t bandwidth, uint8\_\-t symbol\-Dur, uint8\_\-t symbol\-Rate)
\begin{CompactList}\small\item\em Initializing the AGC values. \item\end{CompactList}\item 
void \hyperlink{ntrxiqpar_8h_59949d54cc28e3dd2cbef880a688fb82}{NTRXSet\-Corr\-Threshold} (uint8\_\-t bandwidth, uint8\_\-t symbol\-Dur)
\begin{CompactList}\small\item\em Initializing of correlator threshold values for a specific mode. \item\end{CompactList}\item 
void \hyperlink{ntrxiqpar_8h_8a4de89a7cf28b86ea8d9907f32d0914}{NTRXSet\-Rx\-Iq\-Matrix} (uint8\_\-t bandwidth, uint8\_\-t symbol\-Dur)
\begin{CompactList}\small\item\em Initializing of iq parameter for the receiver. \item\end{CompactList}\item 
void \hyperlink{ntrxiqpar_8h_7f5ee7c54e883b4031423900cf81bfa4}{NTRXSet\-Tx\-Iq\-Matrix} (uint8\_\-t bandwidth, uint8\_\-t symbol\-Dur)
\begin{CompactList}\small\item\em Initializing of iq parameter for the transmitter. \item\end{CompactList}\end{CompactItemize}


\subsection{Detailed Description}
Utility functions for nano\-LOC transmit mode setting. 

\begin{Desc}
\item[Date:]2007-Dez-11 \end{Desc}
\begin{Desc}
\item[Author:]S. Radtke, O. Tekyar {\tt }(C) 2007 Nanotron Technologies\end{Desc}
\begin{Desc}
\item[Note:]Build\-Number = \char`\"{}Build\-Number : 7951\char`\"{};

This file contains the prototypes for setting the transmission modes of the nano\-LOC chip.\end{Desc}
\begin{Desc}
\item[Revision]6839 \end{Desc}
\begin{Desc}
\item[Date]2009-08-13 09:00:45 +0200 (Do, 13 Aug 2009) \end{Desc}
\begin{Desc}
\item[Last\-Changed\-By]sra \end{Desc}
\begin{Desc}
\item[Last\-Changed\-Date]2009-08-13 09:00:45 +0200 (Do, 13 Aug 2009) \end{Desc}


Definition in file \hyperlink{ntrxiqpar_8h-source}{ntrxiqpar.h}.

\subsection{Function Documentation}
\hypertarget{ntrxiqpar_8h_f755eef3bbb9b3f1990d8754679a5695}{
\index{ntrxiqpar.h@{ntrxiqpar.h}!NTRXSetAgcValues@{NTRXSetAgcValues}}
\index{NTRXSetAgcValues@{NTRXSetAgcValues}!ntrxiqpar.h@{ntrxiqpar.h}}
\subsubsection[NTRXSetAgcValues]{\setlength{\rightskip}{0pt plus 5cm}void NTRXSet\-Agc\-Values (uint8\_\-t {\em bandwidth}, uint8\_\-t {\em symbol\-Dur}, uint8\_\-t {\em symbol\-Rate})}}
\label{ntrxiqpar_8h_f755eef3bbb9b3f1990d8754679a5695}


Initializing the AGC values. 

\begin{Desc}
\item[Parameters:]
\begin{description}
\item[{\em bandwidth}]this parameter determines the the mode (fdma or 80 MHz) \item[{\em symbol\-Dur}]this parameter determines the symbol duration between 500ns and 8us) \item[{\em symbol\-Rate}]this parameter determines the symbol rate between 125kbit and 2Mbit)\end{description}
\end{Desc}
This function initializes the agc values for a specific mode. 

Definition at line 500 of file ntrxiqpar.c.\hypertarget{ntrxiqpar_8h_59949d54cc28e3dd2cbef880a688fb82}{
\index{ntrxiqpar.h@{ntrxiqpar.h}!NTRXSetCorrThreshold@{NTRXSetCorrThreshold}}
\index{NTRXSetCorrThreshold@{NTRXSetCorrThreshold}!ntrxiqpar.h@{ntrxiqpar.h}}
\subsubsection[NTRXSetCorrThreshold]{\setlength{\rightskip}{0pt plus 5cm}void NTRXSet\-Corr\-Threshold (uint8\_\-t {\em bandwidth}, uint8\_\-t {\em symbol\-Dur})}}
\label{ntrxiqpar_8h_59949d54cc28e3dd2cbef880a688fb82}


Initializing of correlator threshold values for a specific mode. 

\begin{Desc}
\item[Parameters:]
\begin{description}
\item[{\em bandwidth}]this parameter determines the the mode (fdma or 80 MHz) \item[{\em symbol\-Dur}]this parameter determines the symbol duration between 500ns and 8us)\end{description}
\end{Desc}
This function initializes the correlator threshold registers for a specific mode. 

Definition at line 560 of file ntrxiqpar.c.\hypertarget{ntrxiqpar_8h_8a4de89a7cf28b86ea8d9907f32d0914}{
\index{ntrxiqpar.h@{ntrxiqpar.h}!NTRXSetRxIqMatrix@{NTRXSetRxIqMatrix}}
\index{NTRXSetRxIqMatrix@{NTRXSetRxIqMatrix}!ntrxiqpar.h@{ntrxiqpar.h}}
\subsubsection[NTRXSetRxIqMatrix]{\setlength{\rightskip}{0pt plus 5cm}void NTRXSet\-Rx\-Iq\-Matrix (uint8\_\-t {\em bandwidth}, uint8\_\-t {\em symbol\-Dur})}}
\label{ntrxiqpar_8h_8a4de89a7cf28b86ea8d9907f32d0914}


Initializing of iq parameter for the receiver. 

\begin{Desc}
\item[Parameters:]
\begin{description}
\item[{\em bandwidth}]this parameter determines the the mode (fdma or 80 MHz) \item[{\em symbol\-Dur}]this parameter determines the symbol duration between 500ns and 8us)\end{description}
\end{Desc}
This function initializes the iq parameter for the receiver depending on the bandwidth and symbol duration. 

Definition at line 333 of file ntrxiqpar.c.\hypertarget{ntrxiqpar_8h_7f5ee7c54e883b4031423900cf81bfa4}{
\index{ntrxiqpar.h@{ntrxiqpar.h}!NTRXSetTxIqMatrix@{NTRXSetTxIqMatrix}}
\index{NTRXSetTxIqMatrix@{NTRXSetTxIqMatrix}!ntrxiqpar.h@{ntrxiqpar.h}}
\subsubsection[NTRXSetTxIqMatrix]{\setlength{\rightskip}{0pt plus 5cm}void NTRXSet\-Tx\-Iq\-Matrix (uint8\_\-t {\em bandwidth}, uint8\_\-t {\em symbol\-Dur})}}
\label{ntrxiqpar_8h_7f5ee7c54e883b4031423900cf81bfa4}


Initializing of iq parameter for the transmitter. 

\begin{Desc}
\item[Parameters:]
\begin{description}
\item[{\em bandwidth}]this parameter determines the the mode (fdma or 80 MHz) \item[{\em symbol\-Dur}]this parameter determines the symbol duration between 500ns and 8us)\end{description}
\end{Desc}
This function initializes the iq parameter for the transmitter depending on the bandwidth and symbol duration. 

Definition at line 397 of file ntrxiqpar.c.